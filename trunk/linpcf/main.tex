\documentclass[10pt,a4]{article}

\pagestyle{plain}

%% \setlength{\parskip}{0.3em}
%% \setlength{\textwidth}{170mm}
%% \setlength{\textheight}{240mm}
%% \setlength{\headheight}{0mm}
%% \setlength{\headsep}{0mm}
%% \setlength{\topmargin}{0mm}
%% \setlength{\oddsidemargin}{-5.4mm}
%% \setlength{\evensidemargin}{-5.4mm}

\usepackage{fullpage}
\usepackage{amsmath,amssymb,stmaryrd,latexsym}
\usepackage[bbgreekl]{mathbbol}
\usepackage{graphicx}
\usepackage{proof}
\usepackage{url}
\usepackage[T1]{fontenc}
\usepackage{times}
\usepackage{picinpar}
\usepackage{ifthen}
\usepackage{enumerate}

\newcommand{\pflat}[1]{\bar{#1}}
\newcommand{\defeq}{\stackrel{\mathrm{def}}{=}}
\newcommand{\pdist}[1]{\overline{#1}}


\newtheorem{definition}{Definition}
\newtheorem{theorem}{Theorem}
\newtheorem{lemma}{Lemma}
\newtheorem{proposition}{Proposition}
\newtheorem{corollary}{Corollary}
\newtheorem{example}{Example}
\newenvironment{proof}{{\sc Proof . }}{{\hfill $\square$}}
%% \newcommand{\myqed}{\hfill $\square$}
%% \newcommand{\ped}{\vspace{-1.6em}}
%% \newcommand{\note}[1]{\noindent\fbox{\parbox{\textwidth}{\em #1}}}

\newcommand{\abind}[2]{\mathtt{bind} \ (\, {#1} \,)  \ \mathtt{in} \ {#2}}
\newcommand{\acase}{\mathtt{case}}
\newcommand{\adv}{\mathcal{A}}
\newcommand{\aspecto}[1]{\xrightarrow{\; {#1} \;}}
\newcommand{\asplinear}{\mathfrak{l}}
\newcommand{\aspmodal}{\mathfrak{m}}
\newcommand{\aspnormal}{\mathfrak{n}}
%
\newcommand{\bang}{\,!}
\newcommand{\bit}{\mathtt{B}}
\newcommand{\bitone}{\mathtt{B}_1}
\newcommand{\bitzero}{\mathtt{B}_0}
\newcommand{\bisimul}{\approx}
%
\newcommand{\caseto}{\triangleright}
\newcommand{\cbind}[3]{\mathtt{bind} \ {#1} = {#2} \ \mathtt{in} \ {#3}}
\newcommand{\ccase}[4]{\mathtt{case} \; {#1} \; \mathtt{of} \; \cnil \caseto {#2}; \;
                                          \bitzero x_0 \caseto {#3}; \; \bitone x_1 \caseto {#4}}
\newcommand{\cpred}{\mathtt{pred}}
\newcommand{\csucc}{\mathtt{succ}}
\newcommand{\czero}{\mathtt{zero}}
\newcommand{\ceq}{\simeq}
\newcommand{\cif}[3]{\mathtt{if} \; {#1} \; \mathtt{then} \; {#2} \; \mathtt{else} \; {#3}}
\newcommand{\ctrue}{\mathtt{true}}
\newcommand{\cfalse}{\mathtt{false}}
\newcommand{\cfix}{\mathtt{fix}}
\newcommand{\clet}[3]{\mathtt{let} \ {#1} = {#2} \ \mathtt{in} \ {#3}}
\newcommand{\cnil}{\mathtt{nil}}
\newcommand{\conc}{\! \bullet \!}
\newcommand{\concat}{\pmb{\mathit{conc}}}
\newcommand{\cpair}[1]{\langle{#1}\rangle}
\newcommand{\cproj}{\mathtt{proj}}
\newcommand{\crand}{\mathtt{rand}}
\newcommand{\crec}{\mathtt{rec}}
\newcommand{\crule}[1]{\mbox{{\textsl{#1}}}}
\newcommand{\cslrtype}{\Vdash}
\newcommand{\cut}{\pmb{\mathit{cut}}}
\newcommand{\cval}{\mathtt{val}}
\newcommand{\choice}{\,\square\,}
\newcommand{\context}{\mathbb{C}}
\newcommand{\conteq}{\simeq}
\newcommand{\contpre}{\sqsubseteq^C}
%
\newcommand{\deduces}{\vdash}
\newcommand{\distr}{\mathcal{D}}
\newcommand{\diverge}{\Uparrow}
%
\newcommand{\Eval}[1]{{\left\llbracket{#1}\right\rrbracket}}
\newcommand{\evcontext}{\mathcal{E}}
\newcommand{\expansion}[1]{\tilde{{#1}}}
%
\newcommand{\FV}{\mathit{FV}}
\newcommand{\FLV}{\mathit{FLV}}
\newcommand{\FNV}{\mathit{FNV}}
%
\newcommand{\getval}{\leftarrow}
%
\newcommand{\head}{\pmb{\mathit{hd}}}
\newcommand{\hole}{[\,]}
%
\newcommand{\indist}{\simeq}
\newcommand{\iteration}{\pmb{\mathit{it}}}
%
\newcommand{\ldot}{\, . \, }
\newcommand{\linearto}{\multimap}
\newcommand{\lintransto}[1]{\,\circ\hspace{-0.5em}\xrightarrow{\,{#1}\,}}
\newcommand{\linto}{\multimap}
\newcommand{\logrel}{(\mathcal{R}_\tau)_{\mbox{$\tau$ type}}}
\newcommand{\lr}[1]{\mathcal{R}_{#1}}
%
\newcommand{\metalambda}{\underline{\lambda}}
\newcommand{\modal}{\square}
\newcommand{\mredto}{\hookrightarrow}
\newcommand{\mreduceto}{\downdownarrows}
%
\newcommand{\nat}{\mathbb{N}}
\newcommand{\nullenv}{\varnothing}
\newcommand{\notredto}{\diagup\!\!\!\!\!\!\!\rightsquigarrow}
\newcommand{\notmredto}{\diagup\!\!\!\!\!\!\!\hookrightarrow}
%
\newcommand{\orc}{\mathcal{O}}
\newcommand{\oslrtype}{\Vdash}
\newcommand{\opsimul}{\gtrsim}
\newcommand{\opcontpre}{\succcurlyeq}
%
\newcommand{\prectxt}{\sqsubseteq^C}
\newcommand{\preord}{\sqsubseteq}
\newcommand{\pretrace}{\sqsubseteq^T}
\newcommand{\prefix}{\pmb{\mathit{pref}}}
\newcommand{\proba}{\mathbf{Pr}}
\newcommand{\product}{\,\&\,}
\newcommand{\pchoice}[1]{\,_{#1}\oplus\,}
\newcommand{\PROG}{\mathit{PROG}}
%
\newcommand{\randlen}{\pmb{\mathit{rl}}}
\newcommand{\randnum}{\pmb{\mathit{rn}}}
\newcommand{\randstr}{\pmb{\mathit{rs}}}
\newcommand{\reduce}{\leadsto}
\newcommand{\reduceto}{\rightsquigarrow}
\newcommand{\Reduceto}{\Downarrow}
%
\newcommand{\set}[1]{\{{#1}\}}
\newcommand{\setbits}{\mathbb{B}}
\newcommand{\simul}{\lesssim}
\newcommand{\simuleq}{\sim}
%
\newcommand{\tail}{\pmb{\mathit{tl}}}
\newcommand{\tbits}{\mathsf{Bits}}
\newcommand{\tbool}{\mathsf{Bool}}
\newcommand{\tensor}{\otimes}
\newcommand{\tnat}{\mathsf{Nat}}
\newcommand{\tcomp}{\mathsf{T}}
\newcommand{\ttcase}{\mathtt{case}}
\newcommand{\ttin}{\mathtt{in}}
\newcommand{\ttlet}{\mathtt{let}}
\newcommand{\ttof}{\mathtt{of}}
\newcommand{\tunit}{\mathsf{unit}}
\newcommand{\transto}[1]{\xrightarrow{\,{#1}\,}}
\newcommand{\trace}{\mathcal{T}\!\mathit{r}}
%
\newcommand{\vff}{f\hspace{-0.4em}f} %Boolean value --- false 
\newcommand{\vnil}{\epsilon}
\newcommand{\vtt}{t\hspace{-0.1em}t} %Boolean value --- true
%
\newcommand{\zero}{\mathbf{0}}


\begin{document}

\title{On program equivalence in linear contexts}
\author{Yu Zhang}
\date{Draft of~\today}

\maketitle

\section{Introduction}

Contextual equivalence is an important concept in programming languages and can be used to 
formalize and reason about many interesting properties of computing systems. 
On the other side, linear logic (and its term correspondence often known as linear lambda-calculus) 
has seen significant applications in computer science ever since its birth, 
due to its native mechanism of describing restricted resources. 

Introducing linearity also leads to novel observation over program equivalences. 
In particular, if we consider a special sort of contexts where candidate programs must be used linearly 
(we call these contexts ``linear contexts''), program equivalence w.r.t. these contexts should be a coarser  
relation that the normal contextual equivalence, especially when non-determinism is present. 
For instance, take the language of~\cite{jefferay}, where we have a primive $\choice$ for non-deterministic 
choice (same as the external choice in CSP), and consider the following two functions:
\[
f_1 \defeq \cval(\lambda x \ldot \cval(0) \choice \cval(1)) ,
\qquad
f_2 \defeq \cval(\lambda x \ldot \cval(0)) \choice \cval(\lambda x \ldot \cval(1)).
\]
Existing techniques such as bisimulation~\cite{} or logical relations~\cite{} 
do not capture these two functions. In fact, it is easy to show that these two functions are not 
contextually equivalent, by considering, e.g., the context 
\[
\cbind{f}{[\,]}{} \cbind{x}{f (0)}{}\cbind{y}{f (0)}{\cval(x = y)}
\]
The context makes a double evaluation of the function by applying it to concrete arguments: 
with the first function $f_1$, the two evaluation can return different values since the non-determinstic 
choice is inside the function body; with the second function $f_2$, the non-deterministic choice is made 
before both evaluations and computation inside the function is deterministic, so the two evaluation 
always return the same value. 
But if we consider only linear contexts, where the function will be evaluated no more than once, 
then the two functions should be equivalent. However, no existing technique, at least in our knowledge, 
can establish such an equivalence relation under linear contexts. 

This paper is organized as follows: Section~\ref{sec:linpcf} defines briefly a linear version of PCF 
with a dual type system, as well as its operational semantics. A labeled transition system for the 
language is also presented and the notion of {\em trace equivalence} is defined. 
In section~\ref{sec:lce} we introduce the notion of linear contextual equivalence and show that 
trace equivalence in linear PCF conincides with linear contextual equivalence. 
Section~\ref{sec:monad} extends the result to languages with non-deterministic/probabilistic 
computations and section~\ref{sec:cslr} presents an application of our result in cryptographic proofs. 
Section~\ref{sec:conclu} conludes the paper.

\section{The linear PCF}
\label{sec:linpcf}

We start with a linear version of PCF: types are defined by 
\[
\tau, \tau', \ldots ::= \tnat \mid \tbool \mid \tau \product \tau' \mid \tau \otimes \tau' 
\mid \tau \linto \tau' \mid \tau \to \tau' 
\]
%
$\tau \product \tau'$ and $\tau \tensor \tau'$ are usual product and tensor product types respectively. 
$\tau \linto \tau'$ is the type for linear functions. Following Plotkin~\cite{}, 
we choose to make intuitionistic function types $\tau \to \tau'$ primitive rather than exponential 
types. The choice makes our technical development simpler but does not affect the heart of the proof 
--- one can easily replace it with $\bang$-types and adapt our technique to the new language.

Terms of the linear PCF is defined by 
\[
\begin{array}{lrl@{\qquad}l}
e, e', \ldots 
& ::= & x & 
\\ 
& \mid & 0 \mid 1 \mid 2 \mid \ldots & \mbox{Integers}
\\
& \mid & \csucc \mid \cpred \mid \czero & \mbox{Integer operations}
\\
& \mid & \ctrue \mid \cfalse & \mbox{Booleans}
\\
& \mid & \cif{e_1}{e_2}{e_3} & \mbox{Conditional}
\\ 
& \mid & \cpair{e, e'} \mid \cproj_i(e) & \mbox{Usual products and projections}
\\
& \mid & e \otimes e' \mid \clet{x \otimes y}{e}{e'} & \mbox{Tensor products and projection}
\\
& \mid & \cfix_\tau& \mbox{Fix-point recursion}
\\
& \mid & \lambda x \ldot e \mid e \, e' & \mbox{Abstraction and application}
%% \\
%% & \mid & \cval(e) \mid \cbind{x}{e}{e'} & \mbox{Computational constructs}
%% \\
%% & \mid & e_1 \square e_2 & \mbox{Non-deterministic/probabilistic choice}
\end{array}
\]
%
We write $\PROG(\tau) = \set{e \mid \emptyset ; \emptyset \vdash e : \tau }$ 
for the set of all closed programs of type $\tau$. 
We write $\FV(e), \FLV(e), \FNV(e)$ for the sets of, respectively, free variables, 
free linear varibales, and free non-linear variables in term $e$.

We adopt the type system for dual intuitionistic linear logic~\cite{Barber}: typing judgements 
are of the form $\Gamma; \Delta \vdash e : \tau$, where $\Gamma$ and $\Delta$ are typing environments 
for, respectively, intuitionistic variables and linear variables. 
It is assumed that the codomains of the non-linear and linear typing environments are disjoint.
%
Typing rules:
\[
\begin{array}{c}
\begin{prooftree}
x : \tau \in \Gamma
\justifies 
\Gamma ; \emptyset\vdash x : \tau 
\end{prooftree}
\qquad
\begin{prooftree}
x : \tau \not\in \Gamma
\justifies 
\Gamma ; x : \tau \vdash x : \tau 
\end{prooftree}
\qquad
\begin{prooftree}
\justifies 
\Gamma ; \emptyset \vdash \czero : \tnat 
\end{prooftree}
\qquad
\begin{prooftree}
\justifies 
\Gamma ; \emptyset \vdash \csucc : \tnat \linto \tnat
\end{prooftree}
\\[1.5em]
\begin{prooftree}
\justifies 
\Gamma ; \emptyset \vdash \ctrue : \tbool
\end{prooftree}
\qquad
\begin{prooftree}
\justifies 
\Gamma ; \emptyset \vdash \cfalse : \tbool
\end{prooftree}
\qquad 
\begin{prooftree}
\Gamma; \Delta \vdash e_1 : \tbool
\quad 
\Gamma; \Delta' \vdash e_2 : \tau
\quad 
\Gamma; \Delta' \vdash e_3 : \tau
\justifies 
\Gamma ; \Delta, \Delta' \vdash \cif{e_1}{e_2}{e_3} : \tau
\end{prooftree}
\\[1.5em]
\begin{prooftree}
\Gamma ; \Delta \vdash e_i : \tau_i \; (i = 1, 2)
\justifies
\Gamma ; \Delta \vdash \cpair{e_1, e_2} : \tau_1 \product \tau_2
\end{prooftree}
\qquad
\begin{prooftree}
\Gamma ; \Delta \vdash e : \tau_1 \product \tau_2
\justifies
\Gamma ; \Delta \vdash \cproj_i (e) : \tau_i \; (i = 1, 2)
\end{prooftree}
\\[1.5em]
\begin{prooftree}
\Gamma ; \Delta_i \vdash e_i : \tau_1 \; (i = 1, 2)
\justifies
\Gamma ; \Delta_1, \Delta_2 \vdash e_1 \tensor e_2 : \tau_1 \tensor \tau_2
\end{prooftree}
\qquad 
\begin{prooftree}
\Gamma ; \Delta, x : \tau_1, y : \tau_2 \vdash e : \tau 
\quad
\Gamma ; \Delta' \vdash e' : \tau_1 \tensor \tau_2 
\justifies
\Gamma ; \Delta, \Delta' \vdash \clet{x \tensor y}{e'}{e} : \tau 
\end{prooftree}
\\[1.5em]
\begin{prooftree}
\Gamma, x : \tau ; \Delta  \vdash e : \tau'
\justifies 
\Gamma ; \Delta \vdash \lambda x \ldot e : \tau \to \tau'
\end{prooftree}
\qquad 
\begin{prooftree}
\Gamma ; \Delta  \vdash e : \tau' \to \tau
\quad 
\Gamma ; \emptyset \vdash e' : \tau' 
\justifies 
\Gamma ; \Delta \vdash e \, e'  : \tau'
\end{prooftree}
\\[1.5em]
\begin{prooftree}
\Gamma ; \Delta , x : \tau \vdash e : \tau'
\justifies 
\Gamma ; \Delta \vdash \lambda x \ldot e : \tau \linto \tau'
\end{prooftree}
\qquad 
\begin{prooftree}
\Gamma ; \Delta  \vdash e : \tau' \linto \tau
\quad 
\Gamma ; \Delta'\vdash e' : \tau' 
\justifies 
\Gamma ; \Delta, \Delta' \vdash e \, e'  : \tau'
\end{prooftree}
\qquad 
\prooftree
\justifies
\Gamma ; \emptyset \vdash \cfix_\tau : \tau \to \tau
\endprooftree
%% \\[1.5em]
%% \begin{prooftree}
%% \Gamma ; \Delta \vdash e : \tau 
%% \justifies 
%% \Gamma ; \Delta \vdash \cval(e) : \tcomp \tau 
%% \end{prooftree}
%% \qquad 
%% \begin{prooftree}
%% \Gamma ; \Delta \vdash e_i : \tcomp \tau \; (i = 1, 2)
%% \justifies
%% \Gamma ; \Delta \vdash e_1 \choice e_2 : \tcomp \tau
%% \end{prooftree}
%% \qquad 
%% \begin{prooftree}
%% \Gamma ; \emptyset \vdash e_1 : \tcomp \tau_1  
%% \quad 
%% \Gamma, x: \tau_1 ; \Delta \vdash e_2 : \tcomp \tau_2
%% \justifies 
%% \Gamma ; \Delta \vdash \cbind{x}{e_1}{e_2} : \tcomp \tau_2
%% \end{prooftree}
%% \\[1.5em]
%% \begin{prooftree}
%% \Gamma ; \Delta \vdash e_1 : \tcomp \tau_1  
%% \quad 
%% \Gamma; \Delta' ,  x: \tau_1 \vdash e_2 : \tcomp \tau_2
%% \justifies 
%% \Gamma ; \Delta, \Delta' \vdash \cbind{x}{e_1}{e_2} : \tcomp \tau_2
%% \end{prooftree}
\end{array}
\]

\subsection{The operational semantics}

The operational semantics for linear PCF is a one-step reduction relation $\reduce$ between terms. 
It is indutively defined by the basic reductions
\[
\begin{array}{c}
(\lambda x . e) e' \reduce e [ e' / x ], 
\qquad
\cfix_\tau \, e  \reduce e (\cfix_\tau \, e),
\\
\csucc \, n \reduce n+1 \; (n \in \set{0, 1, 2, \ldots}), 
\qquad
\cpred \, (n+1) \reduce n \; (n \in \set{0, 1, 2, \ldots}),
\\
\czero \, 0 \reduce \ctrue ,
\qquad
\czero \, n \reduce \cfalse \; (n \in \set{1, 2, \ldots}), 
\\
\cif{\ctrue}{e_1}{e_2} \reduce e_1 ,
\qquad
\cif{\cfalse}{e_1}{e_2} \reduce e_2
\\
\cproj_i \cpair{e_1, e_2} \reduce e_i \; (i = 1, 2),
\qquad
\clet{x \tensor y}{e_1 \tensor e_2}{e} \reduce e [e_1 / x, e_2 / y]
%% \\
%% \cbind{x}{\cval(e_1)}{e_2} \reduce e_2 [ e_1 / x ]
%% \\
%% \cbind{x}{e_1}{\cval(x)} \reduce e_1
%% \\
%% \cbind{x}{(\cbind{y}{e_1}{e_2})}{e_3} \reduce \cbind{y}{e_1}{}\cbind{x}{e_2}{e_3}, 
%% \mbox{ if } y \not\in FV(e_3)
%% \\
%% \cbind{x}{e_1 \choice e_2}{e} \reduce 
%% (\cbind{x}{e_1}{e}) \choice (\cbind{x}{e_2}{e})
\end{array}
\]
together with the structural rules 
\[
\prooftree
e_1 \reduce e_2
\justifies 
\evcontext[e_1] \reduce \evcontext[e_2]
\endprooftree
\]
where $\evcontext$ is the evaluation context generated by the grammar
\[
\begin{array}{rl}
\evcontext ::= & \hole \mid \csucc(\evcontext) \mid \cpred(\evcontext) \mid \czero(\evcontext) 
\mid \evcontext \, e \mid \cif{\evcontext}{e_1}{e_2} \mid \cproj_i(\evcontext) 
\mid \clet{x \tensor y}{\evcontext}{e} .
%% \mid \cbind{x}{\evcontext}{e} \mid \evcontext \choice e \mid v \choice \evcontext
\end{array}
\]
We often call a term $\evcontext[x]$ an evaluation context as well, where $x$ is a free variable.

The operational semantics of linear PCF is a deterministic, call-by-name evaluation strategy. 
As in Howe's approach~\cite{},  
We write $e \not\reduce$ when there does not exist a term $e'$ such that $e \reduce e'$.

\begin{lemma}
For every well-typed term $e$, if $e \reduce e'$, then $\FLV(e') = \FLV(e)$.
\end{lemma}

The values of linear PCF have the canonical form:
\[
C ::= \csucc \mid \cpred \mid \czero \mid \ctrue \mid \cfalse \mid n \mid \cfix_\tau 
\mid \cpair{e_1, e_2} \mid e_1 \tensor e_2 \mid \lambda x \ldot e 
%% \mid \cval(e) \mid C \choice C
\]
It is clear that terms in canonical form cannot reduce. The following proposition also 
shows that very closed non-reducible term must be in the canonical form.

\begin{proposition}
If $e$ is a closed term and $e \not\reduce$, then $e$ must be in the canonical form.
\end{proposition}
%
\begin{proof}
We prove by induction on the structure of $e$. Analysis for non-canonical forms:
\begin{itemize}
\item $e \equiv \cif{e'}{e_1}{e_2}$: $e'$ must be close and not reducible (otherwise the whole 
  term can be reduced since $\cif{\hole}{e_1}{e_2}$ is an evaluation context). By induction 
  $e'$ must be canonical, i.e., either $\ctrue$ or $\cfalse$, but in both cases, the original 
  term can be reduced.
\item $e \equiv \cproj_i(e')$: $e'$ must be close and not reducible (since $\cproj_i \hole$ is 
  an evaluation context), and by induction, must be the canonical form $\cpair{e_1, e_2}$, 
  which makes the original term reducible.
\item $e \equiv \clet{x \tensor y}{e'}{e''}$: $e'$ must be close and not reducible, 
  and by induction, must be the canonical form $e_1 \tensor e_2$, which makes the original 
  term reducible.
\item $e \equiv e' \, e''$: $e'$ must be close and not reducible, and by induction, must be 
  canonical: if $e'$ is an abstraction or a fix-point, then the whole term can be reduced; 
  if $e' \in \set{\csucc, \cpred, \czero}$, then $e''$ must be canonical, which will be an 
  integer, hence the whole term can be reduced too.
%% \item $e \equiv \cbind{x}{e'}{e''}$: $e'$ must be close and not reducible, and by induction, 
%%   must be canonical, which can be either $\cval(\_)$ or $\_ \choice \_$. In both cases, 
%%   the original term is reducible.
%% \item $e \equiv e_1 \choice e_2$: apprently both $e_1$ and $e_2$ must be canonical, otherwise 
%%   the term $e$ can be reduced.
\end{itemize}
\end{proof}

%% It is easy to see that a canonical form of a computation type must be of the form 
%% $\cval(e_1) \choice \cdots \choice \cval(e_n)$ ($n \geq 1$).

\subsection{A labeled transition system}

We follow Gordon~\cite{} to define a labelled transition system for the linear PCF. 
As the type system plays an important role in linear PCF, we make the typing of terms 
explicit in trasition rules:
\[
\begin{array}{c}
\prooftree
\ell \in \set{\ctrue, \cfalse, 0, 1, 2, \ldots}
\justifies
\ell \xrightarrow{\,\ell\,} \pmb{\Omega}
\endprooftree
\qquad 
\prooftree
\Gamma; \Delta \vdash \lambda x \ldot e : \tau' \to \tau
\quad 
\Gamma; \emptyset \vdash e' : \tau'
\justifies
\lambda x \ldot e \xrightarrow{@e'} e [e'  / x]
\endprooftree
\\[1.5em]
\prooftree
\Gamma; \Delta \vdash \lambda x \ldot e : \tau' \linto \tau
\quad 
\Gamma; \Delta' \vdash e' : \tau'
\justifies
\lambda x \ldot e \xrightarrow{@e'} e [e'  / x]
\endprooftree
\qquad
\prooftree
\Gamma; \Delta \vdash \cpair{e_1, e_2} : \tau_1 \product \tau_2
\justifies
\cpair{e_1, e_2} \xrightarrow{\cproj_i} e_i
\endprooftree
\\[1.5em]
\prooftree
\Gamma; \Delta \vdash e_1 \tensor e_2 : \tau_1 \tensor \tau_2
\qquad 
\Gamma; \Delta', x: \tau_1, y:\tau_2 \vdash e : \tau
\justifies
e_1 \tensor e_2 \xrightarrow{\tensor e} e [e_1 / x, e_2 / y]
\endprooftree
%% \\[1.5em]
%% \prooftree
%% \Gamma; \emptyset \vdash \cval(e) : \tcomp \tau 
%% \quad
%% \Gamma, x : \tau ; \Delta \vdash e' : \tcomp \tau'
%% \justifies
%% \cval(e) \transto{\tcomp \, e'} e'[e/x]
%% \endprooftree
%% \qquad
%% \prooftree
%% \Gamma; \emptyset \vdash e_i : \tcomp \tau \; (i = 1, 2)
%% \quad 
%% \Gamma, x: \tau; \Delta \vdash e : \tcomp \tau'
%% \justifies
%% e_1 \choice e_2 \transto{\choice e} 
%% (\cbind{x}{e_1}{e}) \choice (\cbind{x}{e_2}{e})
%% \endprooftree
%% \\[1.5em]
%% \prooftree
%% \Gamma; \Delta \vdash \cval(e) : \tcomp \tau 
%% \quad
%% \Gamma; \Delta', x: \tau \vdash e' : \tcomp \tau'
%% \justifies
%% \cval(e) \transto{\tcomp \, e'} e'[e/x]
%% \endprooftree
%% \qquad
%% \prooftree
%% \Gamma; \Delta \vdash e_i : \tcomp \tau \; (i = 1, 2)
%% \quad 
%% \Gamma; \Delta', x: \tau \vdash e : \tcomp \tau'
%% \justifies
%% e_1 \choice e_2 \transto{\choice e} 
%% (\cbind{x}{e_1}{e}) \choice (\cbind{x}{e_2}{e})
%% \endprooftree
\qquad
\begin{prooftree}
e \reduce e'' \qquad e'' \xrightarrow{\,\alpha\,} e'
\justifies 
e \xrightarrow{\,\alpha\,} e'
\end{prooftree}
\end{array}
\]
where $\pmb{\Omega} \defeq \cfix (\lambda x \ldot x)$. 

The last rule says that term reductions are considered as internal transitions ---
external transitions are labeled by {\em actions}. 
Intuitively, external transitions represent the way terms interact with environments. 
For instance, a $\lambda$-abstraction can ``eat'' (application of itself to) a term, 
which is supplied by the environment as an argument, and forms a $\beta$-reduction.
The first rule says that what 
an integer or boolean constant can provide to the environment is the value of itself, 
and after that it can no more provide any information, hence no external transitions 
can occur any more and we represent this by a transition, labeled by the value of 
the constant, into a non-terminating program $\pmb{\Omega}$. 

Let $s$ be a finite sequence of actions $\alpha_1 \alpha_2 \cdots \alpha_n$ ($n \geq 1$). 
We write $e \transto{s}$ if there exist terms $e_1, e_2, \ldots , e_n$ such that 
$e \transto{\alpha_1} e_1 \transto{\alpha_2} e_2 \dots \transto{\alpha_n} e_n$. 
An action sequence $s$ is {\em a trace of $e$} if $e \transto{s}$, and we write 
$\trace(e)$ for the set of all traces of $e$, i.e., $\trace(e) \defeq \set{s \mid e \transto{s}}$.

We define the {\em trace pre-order} $\preord^T$ relation between terms: $e_1 \preord_T e_2$ if 
$\trace(e_1) \subseteq \trace(e_2)$. 
We say two terms $e_1$ and $e_2$ are trace equivalent iff $e_1 \preord^T e_2$ and 
$e_2 \preord^T e_1.$

It is easy to see that if $e \reduce e'$, then $e' \preord^T e$.
%% Some relations on closed terms:
%% \begin{itemize}
%% \item Contextual pre-order $\contpre$: $e_1 \contpre e_2$ if for all context
%% $\context$ such that $\emptyset ; x : \tau \vdash \context : \tcomp \sigma$ holds,
%% $\context[e_1/x] \Reduceto$ implies $\context[e_2/x] \Reduceto$.
%% %
%% %\item Contextual equivalence $\conteq$: $\conteq \defeq \contpre \cap \opcontpre$.
%% %
%% \item Trace pre-order $\preord^T$: $e_1 \preord^T e_2$ if $\trace(e_1) \subseteq \trace (e_2)$.
%% %
%% \item Simulation $\simul$: $e_1 \simul e_2$ implies that for all action $\alpha$, 
%%   if $e_1 \xrightarrow {\, \alpha \,} e_1'$, then there exists $e_2'$ such that 
%%   $e_2 \xrightarrow{\, \alpha \,} e_2'$.
%%   We write $e_1 \opsimul e_2$ if $e_2 \simul e_1$.
%% %
%% \item Simulation equivalence $\simuleq \defeq \simul \cap \opsimul$.
%% %
%% \item Bisimulation $\bisimul$: $e_1 \bisimul e_2$ implies that for all action $\alpha$, 
%% \begin{itemize}
%% \item if $e_1 \transto \alpha e_1'$, then there exists $e_2'$ such that 
%%   $e_2 \transto \alpha e_2'$;
%% \item if $e_2 \transto \alpha e_2'$, then there exists $e_1'$ such that 
%%   $e_1 \transto \alpha e_1'$.
%% \end{itemize}
%% \end{itemize}

\section{Linear contextual equivalence}

Defining a context in a language with linear types must be treated carefully~\cite{Bierman}, 
since holes can hide bound variables and consequently breaks the typing if the variable is linear.
We choose to replace the context hole by an explicit free variable and avoid extra syntactic 
machinery.

Intuitively, a linear context is a context where programs under testing will be evaluated and 
used {\em exactly once}\footnote{It is more general to consider contexts where programs are 
executes {\em at most} once, but clearly a context that does not execute the candidate program 
is of little interest.}.
In a linear functional language, we can formalize it by a restricted 
notion of contexts which contain only one linear free variable and no non-linear variables:
a {\em linear context} $\context_x$ in linear PCF is a program with a single linear variable $x$
and no non-linear variables, i.e., 
$\emptyset; x : \tau \vdash \context_x : \sigma$.
We often omit the variable subscription when it is clear from the texts or irrelevant.

\begin{definition}[Linear contextual equivalence]
We write $e_1 \preord^C_\tau e_2$ for $e_1, e_2 \in \PROG(\tau)$, if 
$\context[e_1 / x] \Reduceto$ implies $\context[e_2 / x] \Reduceto$ for all linear context $\context$ 
such that $\emptyset; x : \tau \vdash \context : \sigma$.
The relation $\preord^C$ is called the {\em linear contextual pre-order} between closed programs. 
{\em Linear contextual equivalence $\simeq^C$} is defined as the symmetrization of $\preord^C$:
$e_1 \simeq^C_\tau$ iff $e_1 \preord^C_\tau e_2$ and $e_2 \preord^C_\tau e_1$.
\end{definition}

Notice that in~\cite{BPR}, the definition of {\em ground contextual equivalence} says that 
contexts must be of an exponential type, but we do not have exponential types. 

\begin{lemma}
Let $\context_1, \context_2$ be two linear contexts such that 
$\emptyset; x: \tau \vdash \context_1 : \sigma$ and $ \emptyset; y : \sigma \vdash \context_2 : \sigma'$, 
then $\context_2[\context_1 / y]$ is also a linear context.
\end{lemma}

Corresponding to the transition system for terms, we also define transitions for linear contexts:
% , which we call {\em linear context transitions}:
\[
\begin{array}{c}
\context[\cif{x}{e_1}{e_2}/y] \lintransto{\ctrue} \context[e_1 / y]
\qquad 
\context[\cif{x}{e_1}{e_2}/y] \lintransto{\cfalse} \context[e_2 / y]
\\[0.2em]
\context[\cpred(x)/y] \lintransto{n} \context[n'/y] \; (n = n' + 1  \mbox{ or } n = n' = 0)
\qquad 
\context[\csucc(x)/y] \lintransto{n} \context[n'/y] \; (n' = n + 1) 
\\[0.2em]
\context[\czero(x)/y] \lintransto{n} \context[\ctrue/y] \; (\mbox{if } n = 0) 
\qquad 
\context[\czero(x)/y] \lintransto{n} \context[\cfalse/y] \; (\mbox{if } n \neq 0) 
\\[0.2em]
\context[\cproj_i(x) / y] \lintransto{\cproj_i} \context_y \; (i = 1, 2)
\qquad
\context[x \, e / y] \lintransto{@e} \context_y 
\qquad
\context[\clet{z_1 \tensor z_2}{x}{e}/y] \lintransto{\tensor e} \context_y
%% \\[0.2em]
%% \context[\cbind{z}{x}{e}/y] \lintransto{\tcomp\,e} \context
%% \qquad
%% \context[\cbind{z}{x}{e}/y] \lintransto{\choice e} \context
\end{array}
\]
%
Linear context transitions represent the way a context interact with programs under testing.
Note that the transition does not necessarily transform a linear context into another linear context ---
linear contexts can also be transformed into closed terms, which do not contain any free varibales.
A linear context transition can eliminate the free variable in the context or transform it into another 
varaible of a different type (in which case we often rename the variable for the sake of clarity), 
but it never introduces new free variables. In particular, in transitions of $@\,e$ and $\tensor\,e$, 
the term $e$ is necessarily closed according to typing. 

According to Lemma~\ref{lem:linvar}, when a linear context can reduce by itself, 
the obtained program is still a linear context.
%
Reductions of linear contexts filled with programs can be classified into several forms, called 
{\em linear context reductions}:
%
\begin{definition}[Linear context reduction]
Let $\context$ be a linear context such that $\emptyset; x :\tau \vdash \context: \sigma$ holds and 
$e$ be a closed program of type $\tau$. 
A reduction of $\context[e / x]$ (if it is reducible) is called a {\em linear context reduction} 
if it is either of the following forms: 
\begin{itemize}
\item $\context [e / x] \reduce \context' [e / x]$, if $\context \reduce \context'$;
\item $\context [e / x] \reduce \context [e' / x]$, if $\context$ is an evaluation context, 
  and $e \reduce e'$;
\item $\context [e / x] \reduce \context' [e' / y]$, if $\context$ is an evaluation context,
  $e \not \reduce$, and $\context_x \lintransto{\alpha} \context_y'$, $e \transto{\alpha} e'$ 
  for some external action $\alpha$.
\end{itemize}
\end{definition}

We shall introduce a so-called {\em linear context lemma}, which is the core of 
proving precongruence of trace pre-order w.r.t. linear contexts. 
The following two lemmas are useful for proving the linear context lemma.

\begin{lemma}
Let $\context_1, \context_2$ be two linear contexts such that 
$\emptyset; x: \tau \vdash \context_1 : \sigma$ and $ \emptyset; y : \sigma \vdash \context_2 : \sigma'$.
If $\context_1 \transto{\alpha} e$, then $\context_2[\context_1/y] \transto{\alpha} \context_2[e/y]$.
\end{lemma}

\begin{lemma}
Let $\context_1, \context_2$ be two linear contexts such that 
$\emptyset; x: \tau \vdash \context_1 : \sigma$ and $ \emptyset; y : \sigma \vdash \context_2 : \sigma'$, 
and $\context_1$ also an evaluation context.
If $\context_2[e/y] \reduce e'$ is a linear context reduction, 
then $\context_1[\context_2[e/y]/x] \reduce \context_1[e'/x]$ is also a linear context reduction. 
\end{lemma}

\begin{lemma}[Linear context lemma]
For every linear context $\context_\tau$ and every closed program $e \in \PROG(\tau)$, if 
$\context[e / x]$ is reducible, then it must be a linear context reduction.
\end{lemma}
%
\begin{proof}
We prove by induction on the structure of the linear context. 

\begin{itemize}
\item $\context$ cannot be any constant since it must contain a linear free variable. And it cannot be 
  a normal product, a tensor product, an abstraction or a trivial computation, as all these forms 
  cannot be reduced any more, no matter what $e$ is.
\item The simplest linear context $x$ is an evaluation context. If $e$ can be reduced, 
  then it is the second case.
\item $\context \equiv \cif{\context'}{e_1}{e_2}$: $\context'$ is another linear context.
  If $\context'[e / x]$ can be reduced, by induction, it must be either of the following cases:
  \begin{itemize}
  \item $\context'[e / x] \reduce \context''[e / x]$ and $\context' \reduce \context''$,
    then $\context[e/x] \reduce \cif{\context''[e/x]}{e_1}{e_2}$ with 
    $\context \reduce \cif{\context''}{e_1}{e_2}$.
  \item $\context'[e/x] \reduce \context'[e'/x]$, $e \reduce e'$, and $\context'$ is an evaluation context, 
    then $\context$ is also an evaluation context, hence $\context[e/x] \reduce \context[e'/x]$.
  \item $\context'[e/x] \reduce \context''[e'/x]$ and $\context' \transto{\alpha} \context''$, 
    $e \transto{\alpha} e'$ for some action $\alpha$, then $\context \transto{\alpha} \cif{C''}{e_1}{e_2}$ 
    and $\context[e / x]$ can take a similar reduction.
  \end{itemize}
  If $\context'[e / x]$ cannot reduce, then it is a canonical boolean term, which is either $\ctrue$ 
  or $\cfalse$, 
  and the only possibility of $\context'$ is the simplest case $x$, with $e$ being a boolear constant. 
  In this case both $\context$ and $e$ can take the transition $\transto{\ctrue}$ or $\transto{\cfalse}$, 
  and the reduction of $\context[e / x]$ falls into the third case.
\item $\context \equiv \cif{e'}{\context_1}{\context_2}$: by typing, both $\context_1$ and $\context_2$ 
  are linear contexts. If $e'$ can be reduced ($e' \reduce e''$), then $\context[e/x]$ will reduce to 
  $\cif{e''}{\context_1[e/x]}{\context_2[e/x]}$, which is still a linear context. If $e'$ cannot be reduced, 
  then it must be a boolean constant since it must be canonical, then $\context[e/x]$ will reduce to 
  either $\context_1[e/x]$ or $\context_2[e/x]$.  Both reductions fit in the first case.
\item $\context \equiv \cproj_i (\context')$: $\context'$ is a linear context. If $\context'[e/x]$ itself 
  can be reudced, then by induction, it must one of the three forms of linear context reduction. 
  In each case, it is easy to see that $\context[e/x]$ will take the same form of reduction. 
  
  If $\context'[e/x]$ is not reducible, then it must be of the form $\cpair{\_, \_}$. There are two cases 
  \begin{itemize}
  \item $\context' \equiv \cpair{\context_1', \context_2'}$, where both $\context_1'$ and $\context_2'$ 
    are linear contexts, then $\context[e/x] \reduce \context_i[e/x]$, which is the second form of 
    linear context reduction.
  \item $\context' \equiv x$ and $e \equiv \cpair{e_1, e_2}$. Now both $\context$ and $e$ can take 
    the transition $\transto{\cproj_i}$: $\context \equiv \cproj_i(x) \transto{\cproj_i} y$,
    $e \equiv \cpair{e_1, e_2} \transto{\cproj_i} e_i$ and $\context[e/x] \reduce e_i = y[e_i/y]$. 
    This is the third form of linear context reduction.
  \end{itemize}
\item $\context \equiv \clet{y_1 \tensor y_2}{\context'}{e'}$: 
  if $\context'$ is reducible, by induction, the reduction of $\context'[e/x]$ must be a linear 
  context reduction, then the reduction of $\context[e/x]$ will be a linear context reduction of 
  the same form.
  
  If $\context'[e/x]$ is not reducible, then there are two cases:
  \begin{itemize}
  \item $\context' \equiv \context'' \tensor e''$ or $\context' \equiv e'' \tensor \context''$,
    where $e''$ is a closed term and $\context''$ is a linear context. 
    Consider the first case without losing generality. $\context[e/x]$ will reduce to 
    $e'[\context''[e/x] / y_1, e'' / y_2]$. It is easy to check that $e'[e''/y_2]$ is also a linear context, 
    then so is $e'[e''/y_2, \context'' / y_1]$, so the reduction is a linear context reduction of 
    the first form.
  \item $\context' \equiv x$ and $e \equiv e_1 \tensor e_2$. Now both $\context$ and $e$ can take a 
    $\transto{\tensor e'}$ transition: 
    $\context \equiv \clet{y_1 \tensor y_2}{x}{e'} \transto{\tensor e'} z$, 
    $e \equiv e_1 \tensor e_2 \transto{\tensor e'} e'[e_1/y_1, e_2/y_2]$, and 
    $\context[e/x] \reduce e'[e_1/y_1, e_2/y_2] = z [e'[e_1/y_1, e_2/y_2]/z]$.
  \end{itemize}
\item $\context \equiv \clet{y \tensor z}{e'}{\context'}$: 
  it is clear that $e'$ is a closed term and $\FLV(\context') = \set{x, y, z}$.
  If $e' \reduce e''$ reduces, then $\context[e/x] \reduce \clet{y \tensor z}{e''}{\context'[e/x]}$. 
  Otherwise, $e'$ must be $e_1' \tensor e_2'$, then 
  $\context[e/x] \reduce \context'[e/x, e_1'/y, e_2'/z]$, with $\context$ reducing to 
  $\context'[e_1'/y, e_2'/z]$, which is a linear context.
\item $\context \equiv \context' \, e'$: 
  Because $\context'$ is a linear context, by induction, if $\context'[e/x]$ can be reduced, then 
  it must be a linear context reduction. As $\context'\,e'$ is an evaluation context, $\context[e/x]$ 
  will take the same form of linear context reduction as $\context'[e'/x]$.

  If $\context'[e/x]$ cannot be reduced, then it must be an abstraction. There are two cases:
  \begin{itemize}
  \item  $\context' \equiv \lambda y \ldot \context''$ and $x \in \FLV(\context'')$, then 
    $\context \equiv (\lambda y \ldot \context'')e' \reduce \context''[e'/y]$ and it is easy to check that 
    $\context''[e'/y]$ is a linear context since $e'$ is closed, 
    hence $\context[e/x] \equiv (\lambda y \ldot \context''[e/x]) e' \reduce \context''[e'/y][e/x]$, 
    which is the first form of linear context reduction. 
  \item $\context' \equiv x$ and $e \equiv \lambda y \ldot e''$ is an abstraction, then 
    $\context \equiv x \, e' \transto{@e'} z$ ($z$ being a fresh linear variable, hence a linear context),
    $e \equiv \lambda y \ldot e'' \transto{@e'} e''[e'/y]$, and 
    $\context[e'/x] \equiv (\lambda y \ldot e'') e' \reduce e''[e'/y] \equiv z[e''[e'/y]/z]$.
  \end{itemize}
\item $\context \equiv e' \, \context'$:  
  If $e' \reduce e''$, then $\context[e/x] \reduce e''(\context'[e/x])$ with 
  $\context \reduce e'' \, \context'$.
  If $e'$ cannot be reduced, then it must be canonical, which is either an abstraction or a constant. 
  Because $\context'$ contains a linear variable, according to the typing system, the type of $e'$ 
  can only be a linear function type. 
  \begin{itemize}
  \item If $e' \equiv \lambda y \ldot e''$, 
    $\context[e/x] \equiv (\lambda y \ldot e'') (\context'[e/x]) \reduce e''[\context'[e/x] / y] 
    = e''[\context'/y][e/x]$. Also $\context \reduce e''[\context'/y]$. 
    Because $y$ is a free linear variable in $e''$, $e''$ is indeed a linear context, so is $e''[\context'/y]$.
  \item If $e'$ is a constant, because its type must be a linear function type, so it can only be one 
    of $\set{\cpred, \csucc, \czero}$. In any case, $e' \, \context'$ is an evaluation context. 
    If $\context'[e/x]$ reduces, then by induction it must be a linear context reduction, hence 
    $e' \, \context[e/x]$ can reduce and is a linear context reduction of the same form as of $\context'[e/x]$.
    If $\context'[e/x]$ cannot reduce, it must be canonical, i.e., an integer $n$, then $\context' \equiv x$ 
    and $e \equiv n$. Now both $\context$ and $e$ can take a $\transto{n}$ transition and 
    $\context[e/x]$ will reduce to another integer or a boolean constant, depending on which constant 
    $e'$ is.
  \end{itemize}
\item $\context \equiv \cbind{y}{\context'}{e'}$:
  This is an evaluation context, so if $\context'[e/x]$ reduces, it must be a linear context reduction, 
  then $\context[e/x]$ is a linear context reduction of the same form as $\context'[e/x]$.
  If $\context'[e/x]$ does not reduce, which must be canonical and of the form 
  $\context_1[e/x] \choice \cdots \choice \context_n[e/x]$, with each $\context_i$ being either 
  $\cval(\context_i')$ (for some $\context_i'$ with free linear variable $x$) or $x$ itself. 
  \begin{itemize}
  \item If $n > 1$, then 
    $\context[e/x] \reduce (\cbind{y}{\context_1[e/x]}{e'}) \choice 
    (\cbind{y}{\context_2[e/x] \choice \cdots \choice \context_n[e/x]}{e'})$,
    with $\context \reduce (\cbind{y}{\context_1}{e'}) \choice 
    (\cbind{y}{\context_2 \choice \cdots \choice \context_n}{e'})$, which is a linear context.
  \item If $n =1$, there are again two cases: 
    \begin{itemize}
    \item If $\context_1 \equiv \cval(\context_1')$, 
      then $\context[e/x] \reduce e'[\context_1'[e/x]/y] = e'[\context_1'/y][e/x]$, with 
      $\context \reduce e'[\context_1'/y]$. Since $\context_1'$ contains linear variables, according to the typing system, 
      $y$ must be a linear variable in $e'$, hence $e'$ is a linear context, and so is $e'[\context_1'/y]$.
    \item If $\context_1 \equiv x$ and $e \equiv \cval(e_1) \choice \cdots \choice \cval(e_m)$. If $m=1$, then both 
      $\context$  and $e$ can take $\transto{\tcomp \, e'}$ transition: 
      $\context \equiv \cbind{y}{x}{e'} \transto{\tcomp \, e'} z$, $e \equiv \cval(e_1) \transto{\tcomp \, e'} e'[e_1/y]$, 
      and $\context[e/x] \equiv \cbind{y}{\cval(e_1)}{e'} \reduce e'[e_1/y] = z[e'[e_1/y]/z]$.
      Similar for the case where $m > 1$.
    \end{itemize}
  \end{itemize}
\item $\context \equiv \cbind{y}{e'}{\context'}$:
  If $e' \reduce e''$, then $\context[e'x] \reduce \cbind{y}{e''}{\context'[e/x]}$ with
  $\context \reduce \cbind{y}{e''}{\context'}$.
  If $e'$ does not reduce, then it must be of the form $\cval(e_1) \choice \cdots \choice \cval(e_n)$. 
  If $n > 1$, then 
  \[
  \context[e/x] \reduce (\cbind{y}{\cval(e_1)}{\context'[e/x]}) \choice 
  (\cbind{y}{\cval(e_2) \choice \cdots \choice \cval(e_n)}{\context'[e/x]})
  \]
  with $\context[e/x] \reduce (\cbind{y}{\cval(e_1)}{\context'}) \choice 
  (\cbind{y}{\cval(e_2) \choice \cdots \choice \cval(e_n)}{\context'})$.
  If $n = 1$, then $\context[e/x] \reduce \context'[e/x][e_1/y] = \context'[e_1/y][e/x]$, 
  with $\context[e/x] \reduce \context'[e_1/y]$, which is a linear context since $e_1$ is closed.
\item $\context \equiv \context_1 \choice \context_2$:  
  Clearly, both $\context_1$ and $\context_2$ are linear contexts. If $\context_1[e/x]$ reduces, then by induction 
  it must be a linear context reduction, then the reduction of $\context[e/x]$ is a linear context reduction of the 
  same form as $\context_1[e/x]$.
  If $\context_1[e/x] \dot \reduce$, then $\context_1[e/x] \choice \context_2$ is an evaluation context, and 
  $\context_2[e/x]$ necessarily reduces, and by induction, the reduction of $\context_2[e/x]$ is a linear context 
  reduction, hence the reduction of $\context[e/x]$ is a linear context reduction of the same form as $\context_2[e/x]$.
\end{itemize}
\end{proof}

\begin{lemma}[Linear context transition lemma]
\label{lem:lc-trans}
Let $\context$ be a linear context with a linear variable $x$ of type $\tau$ and $e$ be a closed program $\tau$, 
such that $\context[e/x] \not\reduce$. A transtion from $\context[e/x]$ must be either of 
the following two forms:
\begin{itemize}
\item $\context[e/x] \transto{\alpha} \context'[e/x]$ with $\context \transto{\alpha} \context'$;
\item $\context \equiv x$ and $\context[e/x] \equiv e \transto{\alpha} e'$.
\end{itemize}
\end{lemma}
%
\begin{proof}
Because $\context[e/x] \not\reduce$, it must be in the canonical form, then $\context$ must be one of the forms:
$x$, $\context_1 \tensor e'$, $e' \tensor \context_1$, $\cpair{context_1, \context_2}$, $\lambda y \ldot \context_1$, 
where $e'$ is a closed term and $\context_1, \context_2$ are linear contexts with free variable $x$.

It is clear that if $\context \equiv x$, then the transition must be of the second form. 
In all other forms, it can be easily checked that the transition will be of the first form, with the context 
$\context$ itself being transformed into another term with (at least) the free variable $x$.
\end{proof}

\begin{theorem}
\label{thm:lc-trace}
Trace pre-order $\preord^T$ is a precongruence with respect to linear contexts, i.e., 
$e_1 \preord^T e_2$ implies that $\context[e_1 / x] \preord^T \context[e_2 / x]$ for all linear contexts $\context$.
\end{theorem}
%
\begin{proof}
According to the definition of $\preord^T$, it sufices to show that, for any action sequence $s$, 
if $\context [e_1 / x] \transto{s}$, then $\context [e_2 / x] \transto{s}$. 
We prove by induction on the length of $\context[e_1 /x ] \transto{s}$ (note that the 
transition includes internal transitions, i.e., term reductions). The base case is trivial.

Wo distinguisha two cases.
\begin{itemize}
\item $\context[e_1 / x] \reduce M \transto{s}$. By the linear context lemma, the reduction must be 
  a linear context reduction, which is one of the following cases:
\begin{itemize}
\item $M \equiv \context'[e_1 / x]$ where $\context \reduce \context'$: it holds that 
  $\context[e_2 / x] \reduce \context'[e_2 / x]$. 
  By induction, $\context'[e_2/x] \transto{s}$ since $\context'[e_1/x] \transto{s}$, 
  hence $\context[e_2/x] \reduce \context'[e_2/x] \transto{s}$, i.e., $\context[e_2/x] \transto{s}$.
\item $M \equiv \context[e_1' / x]$ where $e_1 \reduce e_1'$: 
  we immediately have $e_1' \preord^T e_1 \preord^T e_2$ and by induction, 
  $\context[e_2 / x] \transto{s}$ because $\context[e_1' / x] \transto{s}$.
\item $M \equiv \context_y'[e_1' / y]$ where $\context \lintransto{\alpha} \context_y'$ 
  and $e_1 \transto{\alpha} e_1'$: because $e_1 \preord^T e_2$, $e_2 \transto{\alpha} e_2'$ and 
  $e_1' \preord^T e_2'$. It is clear that $e_2 \transto{\alpha} e_2'$ must be of the form 
  $e_2 \reduce^* e_2'' \transto{\alpha} e_2'$, where $e_2'' \not\reduce$. By the definition of 
  linear context reduction, $\context$ must be an evaluation context, hence 
  $\context[e_2 / x] \reduce^* \context[e_2'' / x] \reduce \context_y'[e_2' / y]$, 
  and by induction, $\context_y'[e_1' / y] \transto{s}$ implies $\context_y'[e_2' / y] \transto{s}$. 
\end{itemize}
\item $\context[e_1 / x] \transto{\alpha} M \transto{s'}$. By Lemma~\ref{lem:lc-trans}, the first transtion 
  has two forms:
\begin{itemize}
\item $\context \transto{\alpha} \context'$ and $M \equiv \context'[e_1/x]$: 
\item $\context \equiv x$ and $e_1 \transto{\alpha} e_1' \equiv M$: $e_1 \preord^T e_2$ implies that 
  $\context[e_2/x] \equiv e_2 \transto{\alpha} \transto{s'}$.
\end{itemize}
\end{itemize}
\end{proof}

\begin{corollary}
Trace equivalence implies linear contextual equivalence.
\end{corollary}

We need also discuss the completeness of trace pre-order w.r.t. linear contextual equivalence. 


\section{Linear contextual equivalence in the computational $\lambda$-calculus}

In this section we consider the language with extensitons of Moggi's computation types~\cite{Moggi}. 
The most interesting cases are non-determonistic and probabilistic computations, as the example 
we have shown in the introduction.

\subsection{Linear PCF with non-determinism}


\subsection{Probabilistic linear PCF}



\section{Applications in cryptography}

This section presents a short proof of the semantic security of EL-Gamal encryption, in CSLR logic~\cite{} 
extended with the rule saying that probabilistic choice can be moved outside a function in a linear context.

\subsection{ASIAN'06 example (non-linear programs)}

In our paper of ASIAN'2006, there is a logical relation for non-determinism monad which fails 
capturing some non-trivial contextually equivalent programs. 
Consider the following two programs:
\[
[\lambda x. [0] \choice [1]]
\qquad 
[\lambda x. [0]] \choice [\lambda x. [0] \choice [1]]
\]
where $\square$ denotes the non-deterministic choice between the two programs to its left and right.
Both programs can be typed as $\tcomp(\tnat \to \tcomp \tnat)$ in Moggi's computational lambda-calculus.

We conjecture that the two programs are contextually equivalent but did not prove it. 
The logical relation for non-deterministic monad is defined as 
\[
c_1 \;\mathcal{R}_{\tcomp \tau}\; c_2 
\Leftrightarrow 
\forall a \in c_1, \exists b \in c_2 \mbox{ s.t. } a \;\mathcal{R}_{\tau}\; b ,
\mbox{ and } 
\forall b' \in c_2, \exists a' \in c_1 \mbox{ s.t. } a' \;\mathcal{R}_{\tau}\; b' .
\]
This logical relation fails relating the above two programs, because the branch $\lambda x. [0]$ 
in the second program has no relating candidate in the first program!

%% Note that the following two programs are not contextually equivalent:
%% \[
%% [\lambda x. [0] \choice [1]]
%% \qquad 
%% [\lambda x. [0]] \choice [\lambda x. [1]] .
%% \]
%% The context 
%% \[
%% \clet{f}{\_}{} \clet{x}{f(0)}{} \clet{y}{f(1)}{[x+y]}
%% \]
%% can distinguish: 
%% for the first program, the possible results are $0, 1, 2$, while for the second, only $0, 2$.

The two programs are not bisimilar either: while both can take a $\oplus$ transition and become
\[
[\lambda x. [0] \choice [1]] \xrightarrow{\choice} \lambda x. [0] \choice [1] 
\xrightarrow{\;@x\;} [0] \choice [1],
\qquad 
[\lambda x. [0]] \choice [\lambda x. [0] \choice [1]] \xrightarrow{\choice} \lambda x. [0]
\xrightarrow{\;@x\;} [0] ,
\]
$[0] \choice [1]$ and $[0]$ are clearly not bisimilar.

The two programs are simulation equivalent.

%% {\bf Conjecture:} Contextual equivalence in the non-deterministic computational lambda-calculus is 
%% a simulation equivalence. 

%% Proof of $\contpre \subseteq \simul$: Suppose that $e_1 \contpre e_2$, i.e., if $e_1 \reduceto$ 
%% then $e_2 \reduceto$. We need to show that $\contpre$ is a simulation, i.e., for every action 
%% $\alpha$, if $e_1 \transto \alpha e_1'$, then there exists $e_2'$ such that 
%% $e_2 \transto \alpha e_2'$ and $e_1' \contpre e_2'$.

%% \begin{itemize}
%% \item The case where $\alpha = @e$ (the cases of $\ell$ and $\cproj_i$ are similar): Notice that 
%% if $e_1 \contpre e_2$, then they must be of the same type, hence both $e_1$ and $e_2$ are of type 
%% $\tau \to \tau'$ for some $\tau, \tau'$, and $e$ has type $\tau$. According to the transition rules, 
%% $e_2$ can take the transition $e_2 \transto{@e} e_2 e$. Assume that $\neg (e_1 e \contpre e_2 e)$, 
%% i.e., there exists some context $\context$ such that $\context\{e_1e/x\} \reduceto$ 
%% but $\context \{e_2 e/ x\} \diverge$. Then we can construct another context 
%% $\context' = \context\{xe/ x \}$. Clearly $\context'\{e_i/x\} = \context\{e_ie/x\}$, and it is 
%% a contradiction to $e_1 \contpre e_2$.
%% %
%% \item The case where $\alpha = \oplus$: First, $e_1$ and $e_2$ must have type $\tcomp \tau$ for 
%% some $\tau$, so if $e_2$ is not a $\reduce$-normal form, it can take several $\reduce$-reductions 
%% and becomes a normal form $\cval(e_{21}) \choice \cdots \choice \cval(e_{2n})$ (if $\reduce$-reduction 
%% does not terminate, then it is easy to show that $e_1 \contpre e_2$ when $e_1$ can take 
%% a $\mredto$-reduction), hence $e_2$ can always take the $\oplus$ transition whenever $e_1$ can.

%% Assume that for every transition $e_2 \transto{\oplus} e_{2i}$, it holds $\neg(e_1' \contpre e_{2i})$.
%% \end{itemize}

\section{Conclusion}


\end{document}
